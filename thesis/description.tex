\section{Апісанне электроннай бібліятэкі}

\subsection{Агульнае апісанне электроннай бібліятэкі}

Электронная бібліятэка --- спарадкаваная калекцыя разнастайных электронных дакументаў (у тым ліку кніг, часопісаў), забяспечаных сродкамі навігацыі і пошуку. Можа быць вэб-сайтам, дзе паступова назапашваюцца розныя тэксты (часцей літаратурныя, але таксама навуковыя і любыя іншыя) і медыяфайлы, кожны з якіх самадастатковы і ў любы момант можа быць запатрабаваны чытачом. Электронныя бібліятэкі могуць быць універсальнымі, якія імкнуцца да найбольш шырокага выбару матэрыялу, і больш спецыялізаванымі.

Электронныя бібліятэкі варта адрозніваць ад сумежных структурных тыпаў сайта, асабліва літаратурнага. У адрозненне ад літаратурнага часопіса, які нарадзіўся як тып друкаванага выдання, але паспяхова і без прынцыповых зменаў структуры перабраўся ў Інтэрнэт, электронная бібліятэка не падзяляецца на выпускі і абнаўляецца перманентна па меры з'яўлення новых матэрыялаў. У адрозненне ад сайта з вольнай публікацыяй, электронная бібліятэка, як правіла, падбіраецца каардынатарам праекта па сваім меркаванні і, што значна больш важна, не прадугледжвае стварэння вакол публікаваных тэкстаў камунікатыўнага асяроддзя.

Электронныя бібліятэкі выконваюць наступныя функцыі:
\begin{enumerate}
    \item інфармацыйная, накіраваная на задавальненне патрэбы ў інфармацыі розных катэгорый карыстальнікаў па ўсіх галінах ведаў;
    \item асветная, якая рэалізуецца, у тым ліку за кошт папулярызацыі электронных дакументаў, якія адносяцца да гісторыі і культуры;
    \item навукова-даследчая, арыентаваная на садзейнічанне глыбокаму вывучэнню тэмы (прадмета) навуковымі работнікамі і спецыялістамі, у тым ліку за кошт прадастаўлення поўных тэкстаў з аддаленых фондаў;
    \item адукацыйная, у рамках якой ажыццяўляецца падтрымка, як асноўнай, так і дадатковай адукацыі шляхам прадастаўлення не толькі мультымедыйнага вучэбнага матэрыялу, але і неабходнай літаратуры;
    \item даведачная, якая дазваляе атрымліваць дакладныя звесткі, адлюстраваныя ў дакументах пэўнага віду;
    \item функцыя захавання творчай спадчыны, асабліва важная ва ўмовах электроннага асяроддзя.
\end{enumerate}

\subsection{Пастаноўка задач да вэб-праекта}

Падчас планавання структуры і мэт вэб-праекта <<электронная бібліятэка>> былі пастаўлены
наступныя патрабаванні:
\begin{enumerate}
    \item вэб-сайт мае забяспечваць магчымасць атрымання pdf версій кніг;
    \item дабаўлення новай літаратуры мае ажыццяўляцца без змянення зыходнага коду праекта;
    \item вэб-сайт мае забяспечваць пошук літаратуры па назве твору;
    \item вэб-сайт мае забяспечваць пошук літаратуры па імю аўтара;
    \item вэб-сайт мае забяспечваць магчымасць рэгістрацыі чытачоў (для магчымасці ў далейшым
          развіцці электроннай бібліятэкі стварыць платны раздзел для пэўных чытачоў, а таксама
          рэалізаваць рассылку паступленняў). Дазваляецца рэгістраваць толькі ўнікальныя імёны чытачоў;
    \item даступнасць вэб-сайта ў Інтэрнэце.
\end{enumerate}

На малюнку \ref{img: tasks} прадстаўлена дыяграма дзеянняў, якія маюць быць забяспечаны чытачу.

\begin{figure}[h!]
    \centering
    \includegraphics[width=\textwidth]{tasks}
    \vspace{-2.5\baselineskip}
    \caption{Дыяграма варыянтаў дзеяння чытача}
    \label{img: tasks} 
\end{figure}
