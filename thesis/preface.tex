\sectionWithoutNumber{\prefacename}

Адной з найважнейшых задач, якія практычна заўсёды стаялі перад
чалавецтвам, з'яўляецца захаванне інфармацыі ў часе і прасторы. Пасля ўзнікнення кнігадрукавання асноўнай формай
фіксацыі распаўсюджвання інфармацыі з'яўляюцца друкаваныя выданні, а
галоўнымі сродкамі захоўвання і доступу да інфармацыі сталі
бібліятэкі. Захаванне і выкарыстанне рукапісных і друкаваных дакументаў дастаткова добра асвоена, тут маюцца багаты вопыт
даследчай і практычнай працы шматлікіх пакаленняў
спецыялістаў. Але відавочна, што аб'ёмы інфармацыі, якая захоўваецца ў
традыцыйнай форме, робяць усё больш цяжкай працу з ёй:
захоўванне, распаўсюджванне, пошукі, ўлік і да т. п..
Развіццё
вылічальнай тэхнікі дазволіла захоўваць і распаўсюджваць
інфармацыю ў электроннай форме, што гуляе рэвалюцыйную ролю ў гісторыі аналагічную вынаходству кнігадрукавання.

Электронная форма дазваляе на сёння захоўваць найбольш надзейна і кампактна, распаўсюджваць яе нашмат больш аператыўна і шырэй і, акрамя таго, дае магчымасці маніпулявання з ёй, якіх не магло быць пры іншых формах. У сувязі з гэтым за апошнія гады ва ўсім свеце інтэнсіўна павялічваецца колькасць электронных публікацый. Значная колькасць розных дакументаў ужо цяпер існуюць у электроннай форме.

Таму забеспячэнне публічнага (у тым ліку аддаленага) доступу карыстальнікаў да электронных інфармацыйных рэсурсаў стала адной з першачарговых задач інфармацыйнага абслугоўвання навукі, адукацыі і культуры. Існуючая зараз сістэма інфармацыйнага абслугоўвання дазваляе нам атрымаць доступ практычна да любой інфармацыі ў неабмежаванай колькасці, велізарную ролю ў гэтым гуляе сетка Internet, якая з 1991 паспела ператварыцца ў самабытнае і практычна невычэрпнае інфармацыйнае асяроддзе. Усе сучасныя сістэмы бібліятэчнага абслугоўвання будуць заснаваныя на прадастаўленні інфармацыі карыстальнікам, з дапамогай «сусветнай павуціны».

У пэўным сэнсе тэрмін электронная бібліятэка вядомы ўжо даўно --- на працягу шэрагу гадоў розныя спецыялісты прадказвалі наступ эры электронных бібліятэк. Разам з тым гэты тэрмін можна лічыць і цалкам новым. Традыцыйныя бібліятэкі з'яўляюцца не толькі калекцыяй дакументаў або крыніц інфармацыі, але маюць таксама сродкі навігацыі, сістэмы каталогаў, службы дапамогі чытачу і аўтаматызаванае асяроддзе працы з інфармацыяй, то прыйдзецца прызнаць, што электронная бібліятэка пачынае набываць некаторыя новыя рысы.

Мэтай курсавой работы з'яўляецца распрацоўка інфраструктуры для захоўвання электронных выданняў кніг і
забеспячэння чытачам доступу да гэтых выданняў пры дапамозе вэб-сайта <<Электронная бібліятэка>>.
