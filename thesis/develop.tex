\section{Распрацоўка электроннай бібліятэкі}

Для напісання вэб-сайта электроннай бібліятэкі быў абраны фрэймворк Flask,
напісаны на мове праграмавання Python.

Дадзены выбар абумоўлены наступнымі прычынамі:
\begin{enumerate}
    \item веданне распрацоўшчыкам мовы праграмавання Python;
    \item распаўсюджанасць мовы праграмавання Python;
    \item прастата напісання вэб-сайтаў на Flask;
    \item багатыя магчымасці ў пабочных бібліятэках Python;
\end{enumerate}

Flask --- фрэймворк для стварэння вэб-праграм на мове праграмавання Python, які выкарыстоўвае набор інструментаў Werkzeug, а таксама шабланізатар Jinja2. Адносіцца да катэгорыі так званых мікрафрэймворкаў --- мінімалістычны каркас вэб-праграм, якія прадастаўляюць толькі базавыя магчымасці.

Flask забяспечвае наступныя функцыі:
\begin{enumerate}
    \item маршрутызацыя;
    \item статычныя файлы;
    \item jinja2 шаблоны;
    \item доступ да параметраў запыту;
    \item перанакіраванне;
    \item уласныя памылкі;
    \item логі;
\end{enumerate}

У лістынгу \ref{lst: main.py} прадстаўлены зыходны код, які забяспечваю логіку вэб-сайта электроннай
бібліятэкі.

\lstinputlisting[caption={Зыходны код логікі вэб-сайта},%
                            label={lst: main.py},%
                            language=Python]{main.py}

Запіс \textit{@app.route("/", methods =['GET', 'POST'])} з'яўляецца дэкаратам, які ўказвае
адрас старонкі, пры пераходзе на якую выклікаецца функцыя, а таксама метады, якія могуць
быць апрацованыя гэтай функцыяй.

Заўважым, што функцыі вяртаюць карыстальнікам не статычную html-старонку,
але апрацаваны шаблон, у які былі падстаўлены перададзеныя пераменныя
(напрыклад, запіс \textit{return render\_template('catalog.html ', books\_info=books\_info)}).
Гэта дазваляе ствараць дынамічныя вэб-старонкі, кантэкст каторай будзе брацца са сховішча
даных у залежнасці ад апісанай логікі.

Jinja --- гэта шабланізатар для мовы праграмавання Python. Ён падобны да шабланізатару Django, але падае Python-падобныя выразы, забяспечваючы выкананне шаблонаў ў пясочніцы. Гэта тэкставы шабланізатар, таму ён можа быць выкарыстаны для стварэння любога віду разметкі, а таксама зыходнага кода. Ліцэнзаваны пад BSD ліцэнзіяй.

Шабланізатар Jinja дазваляе настрайваць тэгі, фільтры, тэсты і глабальныя пераменныя. Таксама, у адрозненні ад шабланізатара Django, Jinja дазваляе канструктару шаблонаў выклікаць функцыі з аргументамі на аб'ектах.

У лістынгу \ref{lst: jinja2 sample} прадстаўлена частка jinja2 шаблона, у якім
падстаўляецца html-код на кнігі, інфармацыя пра якія была атрыманая з базы даных.


\lstinputlisting[caption={Прыклад jinja2 шаблона},%
                            label={lst: jinja2 sample},%
                            language=html]{sample.jinja2}

У лістынгу \ref{lst: jinja2 sample} html-элемент, які апісвае адну кнігу,
знаходзіцца ўнутры 2 цыклаў \textit{for}, якія забяспечваюць дубліраванне
дадзенага html-элемента неабходную колькасць разоў і размяшчэнне яго
ў колькасці 3 кніг на 1 радок.
