\sectionWithoutNumber{Заключэнне}

У выніку выканання дадзенай курсавой работы была выкананая распрацоўка вэб-сайта <<Электронная бібліятэка>>,
які прадастаўляе доступ да электронных выданняў, якія захоўваюцца ў воблачным сервісе Amazon S3.
Вэб-сайт <<Электронная бібліятэка>> быў створаны пры дапамозе дынамічных html-старонак, што дазваляе
дабаўляць новыя літаратурныя выданні без змянення зыходнага коду вэб-сайта.

Дадзены праект прызначаны для захоўвання і прадастаўлення шырокаму колу чытачоў электронных выданняў
знакамітых твораў, навуковых прац.

На дадзены момант электронная бібліятэка прадастаўляе базавы функцыянал, які задавальняе ўсім
пастаўленым патрабаванням.
У якасці далейшых перспект прапануецца:
\begin{enumerate}
    \item дабаўленнe магчымасці аўтэнтыфікацыі чытачоў (для прадастаўленню платнага матэрыялу і яго ўліку);
    \item стварэнне мабільнай версіі праграмы;
    \item дабаўленнe рассылкі зарэгістраваным чытачам.
\end{enumerate}
